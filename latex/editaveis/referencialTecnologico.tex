\chapter[Referencial Tecnológico]{Referencial Tecnológico}
\addcontentsline{toc}{chapter}{Referencial Tecnológico}
\label{chap:tecnologico}

Neste capítulo, serão apresentadas as principais ferramentas e tecnologias que estão sendo utilizadas para elaboração, condução, desenvolvimento e análise do trabalho como um todo. Começando pelas principais ferramentas, as quais viabilizam o desenvolvimento e a análise do projeto, \hyperref[sec:principal]{Ferramentas Associadas ao projeto}, tem-se que: o versionamento/hospedagem com uso do GitHub; 

Entretanto, ainda existem outros apoios tecnológicos de relevâncias, e que merecem ser destacados. Em \hyperref[sec:demais]{Outros Apoios Tecnológicos}, há destaque para Ferramentas de Comunicação (ex. Teams e Whatsapp); e Ferramentas de Escrita da Monografia 
(ex. Latex e Overleaf).

\section{Ferramentas Associadas ao Projeto}
\label{sec:principal}

Seguem as principais ferramentas que serão consumidas para viabilizar o desenvolvimento e a análise do projeto.

\subsection{Python}

Python é uma linguagem de programação de alto nível, interpretada e de propósito geral, conhecida por sua simplicidade e legibilidade. Criada por Guido van Rossum, Ela foi concebida com ênfase na facilidade de uso e na capacidade de expressar ideias de forma concisa. Seu design modular e sintaxe clara a torna uma escolha popular tanto para iniciantes quanto para desenvolvedores experientes. A filosofia central de Python, conhecida como "Zen of Python," destaca princípios como clareza, simplicidade e praticidade, guiando o desenvolvimento da linguagem.

A linguagem suporta diversos paradigmas de programação, sendo mais notavelmente associado à programação orientada a objetos. No entanto, a linguagem é multifacetada, permitindo abordagens imperativas, funcionais e procedural. Essa flexibilidade permite aos desenvolvedores escolherem o paradigma que melhor se adapta ao problema em questão, contribuindo para a versatilidade da linguagem em diferentes contextos de desenvolvimento.

Em relação aos seus prós, Python é reconhecido pela sua legibilidade e sintaxe limpa, fatores que facilitam a manutenção e colaboração em projetos. A vasta biblioteca padrão e a extensa comunidade de desenvolvedores oferecem suporte para uma ampla gama de tarefas, desde desenvolvimento web até análise de dados.

A sintaxe clara e a legibilidade do Python tornam o desenvolvimento de algoritmos de IA mais acessível, permitindo que os desenvolvedores foquem na lógica e nos conceitos subjacentes, em vez de se perderem em detalhes complexos de implementação. Além disso, a vasta biblioteca padrão e o ecossistema robusto de bibliotecas de terceiros, como NumPy, Pandas e Scikit-learn, oferecem ferramentas poderosas para manipulação de dados, processamento matemático e implementação eficiente de algoritmos de aprendizado de máquina.

Outro ponto crucial é a popularidade e o suporte contínuo que Python recebe na comunidade de ciência de dados e IA. A abundância de frameworks e bibliotecas especializadas, como TensorFlow, PyTorch e scikit-learn, todos com interfaces amigáveis, consolida a posição da linguagem como uma escolha estratégica para desenvolvedores e pesquisadores em IA. A combinação de acessibilidade, riqueza de ferramentas e apoio da comunidade faz de Python uma linguagem ideal para a concepção, prototipagem e implementação eficiente de soluções de inteligência artificial.

\subsection{TensorFlow}

TensorFlow é uma poderosa biblioteca de código aberto desenvolvida pelo Google, projetada para facilitar a implementação de algoritmos de aprendizado de máquina e redes neurais. Destaca-se como uma das ferramentas mais populares e abrangentes no campo da inteligência artificial (IA) e aprendizado de máquina (ML). Seu nome deriva da estrutura subjacente de dados que representa fluxos multidimensionais, essenciais para operações matemáticas eficientes, fundamentais no contexto do aprendizado profundo.

A principal força do TensorFlow reside em sua flexibilidade e suporte para diversas plataformas, possibilitando o desenvolvimento e treinamento de modelos em diferentes ambientes, desde dispositivos móveis até sistemas distribuídos em larga escala. Adotando uma abordagem simbólica, o TensorFlow permite que os desenvolvedores definam e otimizem modelos complexos de maneira eficiente, com ênfase especial em redes neurais profundas. Essa flexibilidade é complementada por uma comunidade ativa e uma vasta documentação, facilitando a curva de aprendizado para novos usuários.

Entre os pontos positivos do TensorFlow, destaca-se a sua escalabilidade e eficácia no treinamento de modelos em grandes conjuntos de dados. Sua integração com unidades de processamento gráfico (GPUs) e unidades de processamento tensorial (TPUs) acelera o processamento, tornando-o ideal para projetos de aprendizado profundo.

\subsection{Keras}

Keras é uma poderosa API de alto nível projetada para simplificar o desenvolvimento de modelos de redes neurais, integrando-se perfeitamente ao TensorFlow, sendo frequentemente utilizada em conjunto com essa biblioteca. Criada com o objetivo de tornar o desenvolvimento de modelos de aprendizado profundo mais acessível, Keras oferece uma interface intuitiva e amigável, que abstrai grande parte da complexidade inerente ao aprendizado de máquina, permitindo que desenvolvedores e pesquisadores criem, treinem e testem modelos de forma rápida e eficiente.

Keras desempenha um papel de facilitar a criação e a customização de redes neurais que podem ser adaptadas tanto para o treinamento centralizado quanto para o treinamento distribuído. Sua integração com o \textit{TensorFlow Federated} permite que modelos Keras sejam aplicados em ambientes federados, onde o aprendizado ocorre de maneira descentralizada, mantendo a privacidade dos dados. Isso é particularmente relevante para o treinamento de modelos em sistemas de dados distribuídos, como em cenários onde os dados sensíveis são mantidos nos dispositivos dos usuários.

A simplicidade e modularidade do Keras o tornam uma ferramenta ideal para quem está começando no aprendizado de máquina, ao mesmo tempo em que oferece flexibilidade suficiente para projetos mais avançados. A API permite a construção rápida de modelos sequenciais e funcionais, suportando camadas como convolucionais, totalmente conectadas e recorrentes, as quais são essenciais para resolver problemas complexos em visão computacional, processamento de linguagem natural e outras áreas de IA.

Além disso, Keras oferece suporte para diversas otimizações e métricas que são fundamentais para a análise de desempenho dos modelos, seja em um ambiente centralizado ou federado. Sua capacidade de compilar modelos com diferentes otimizadores e funções de perda, como o \textit{Adam} e a \textit{sparse categorical crossentropy}, permite ajustar o comportamento dos modelos conforme as necessidades específicas do experimento, como no caso do estudo comparativo entre aprendizado centralizado e federado.

Assim como o TensorFlow, o Keras também se beneficia de uma comunidade ativa, rica documentação e suporte para execução em GPUs e TPUs, tornando-o adequado tanto para ambientes de pesquisa quanto para aplicações de produção em larga escala. Isso reforça a importância de sua escolha em projetos que buscam escalabilidade e eficiência, como os que envolvem aprendizado federado, onde o objetivo é treinar modelos em dispositivos descentralizados sem comprometer a privacidade dos dados.

\subsection{Google Colab}

O Google Colab é uma plataforma baseada em nuvem que oferece um ambiente interativo para o desenvolvimento e execução de código Python, especialmente voltado para o aprendizado de máquina e análise de dados. A ferramenta proporciona a capacidade de criar e compartilhar notebooks Jupyter, que permitem a integração de código executável, visualizações e texto descritivo em um único documento. O Google Colab é amplamente valorizado por sua acessibilidade, uma vez que não requer configuração local e oferece acesso a recursos computacionais avançados, como unidades de processamento gráfico (GPUs) e unidades de processamento tensorial (TPUs), sem custo adicional.

No contexto deste projeto, o Google Colab foi utilizado para implementar e testar o modelo de aprendizado federado. A plataforma facilitou a execução do código de maneira eficiente e a utilização de GPUs para acelerar o treinamento do modelo, o que foi crucial para lidar com a complexidade do aprendizado federado e o processamento de grandes conjuntos de dados. Além disso, o Google Colab permitiu a documentação do progresso por meio de notebooks, que incluíram não apenas o código e os resultados das execuções, mas também as análises e visualizações das métricas de desempenho do modelo. A possibilidade de compartilhar facilmente os notebooks com os orientadores e colegas foi um aspecto importante para a colaboração e revisão contínua do projeto.

\subsection{Base de dados Fashion-MNIST}

A base de dados Fashion-MNIST foi desenvolvida como uma alternativa mais complexa ao tradicional MNIST, visando representar um cenário mais próximo de aplicações reais em visão computacional. Composta por imagens de 28x28 pixels em tons de cinza, ela contém 70.000 imagens divididas em 10 classes, todas relacionadas a itens de vestuário, como camisetas, sapatos, bolsas e casacos. Cada classe possui uma variabilidade significativa em termos de estilo e design, o que desafia os modelos de aprendizado de máquina a reconhecer padrões sutis entre os diferentes itens.

No contexto do aprendizado federado, que é o foco principal do projeto, o Fashion-MNIST oferece uma excelente oportunidade para testar a robustez de modelos distribuídos na identificação de padrões em dados não centralizados. Como os dados de usuários de dispositivos móveis, por exemplo, estão frequentemente dispersos e não podem ser agregados facilmente por questões de privacidade, a variabilidade dos dados na base Fashion-MNIST simula um ambiente similar. Cada dispositivo ou cliente pode treinar um subconjunto desses dados, representando diferentes distribuições de moda, para então colaborar no aprimoramento de um modelo global.

\section{Outros Apoios Tecnológicos}
\label{sec:demais}

Seguem as demais ferramentas que auxiliam na na comunicação entre os envolvidos no projeto, bem como na escrita dessa monografia.

\subsection{Ferramentas de Comunicação}

Para comunicação entre autor e orientadores, durante a elaboração desse trabalho, destacam-se: 

\begin{itemize}
	\item O Microsoft Teams, para reuniões virtuais sobre o andamento do trabalho, e 
	\item O Whatsapp, para trocas de mensagens e avisos rápidos.
\end{itemize}

\subsection{Ferramentas de Escrita da Monografia}

Para escrita da monografia, destaca-ase o uso do LaTeX, sendo este uma linguagem de marcação para escrita de monografias comumente utilizada. Como principal vantagem tem-se a escrita 
em forma de texto simples, sem haver tanta preocupação com a formatação durante a escrita do texto. Adicionalmente foi utilizado o Overleaf para escrita do texto no formato LaTeX 
e organização das pastas, além de ser utilizado para compilar a monografia em formato PDF.

\subsection{GitHub}

O GitHub é uma plataforma amplamente utilizada para o versionamento de código e colaboração em projetos de software. Baseado no sistema de controle de versão Git, o GitHub oferece uma interface web que permite o gerenciamento de repositórios de código, acompanhamento de mudanças e coordenação entre equipes de desenvolvimento. A plataforma proporciona funcionalidades essenciais como a criação de branches para desenvolvimento paralelo, pull requests para revisão de código e issues para rastreamento de bugs e tarefas. 

No contexto desta pesquisa, o GitHub foi empregado para organizar e versionar o código-fonte desenvolvido, facilitando a colaboração e a documentação do progresso do projeto. A utilização do GitHub permitiu o controle de versões do código, a integração de novas funcionalidades e a correção de eventuais falhas de forma sistemática e rastreável. Além disso, a plataforma facilitou a colaboração e comunicação com orientadores e colegas, assegurando uma gestão eficiente do desenvolvimento do projeto e garantindo a integridade e a consistência do código ao longo das diferentes etapas do trabalho.
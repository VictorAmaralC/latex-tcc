\chapter[Referencial Tecnológico]{Referencial Tecnológico}
\addcontentsline{toc}{chapter}{Referencial Tecnológico}
\label{chap:tecnologico}

Neste capítulo, serão apresentadas as principais ferramentas e tecnologias que estão sendo utilizadas para elaboração, condução, desenvolvimento e análise do trabalho como um todo. Começando pelas principais ferramentas, as quais viabilizam o desenvolvimento e a análise do projeto, \hyperref[sec:principal]{Ferramentas Associadas ao projeto}, tem-se que: o versionamento/hospedagem com uso do GitHub; 

Entretanto, ainda existem outros apoios tecnológicos de relevâncias, e que merecem ser destacados. Em \hyperref[sec:demais]{Outros Apoios Tecnológicos}, há destaque para Ferramentas de Comunicação (ex. Teams e Whatsapp); e Ferramentas de Escrita da Monografia 
(ex. Latex e Overleaf).

\section{Ferramentas Associadas ao Projeto}
\label{sec:principal}

Seguem as principais ferramentas que serão consumidas para viabilizar o desenvolvimento e a análise do projeto.

\subsection{Python}

Python é uma linguagem de programação de alto nível, interpretada e de propósito geral, conhecida por sua simplicidade e legibilidade. Criada por Guido van Rossum, Ela foi concebida com ênfase na facilidade de uso e na capacidade de expressar ideias de forma concisa. Seu design modular e sintaxe clara a torna uma escolha popular tanto para iniciantes quanto para desenvolvedores experientes. A filosofia central de Python, conhecida como "Zen of Python," destaca princípios como clareza, simplicidade e praticidade, guiando o desenvolvimento da linguagem.

A linguagem suporta diversos paradigmas de programação, sendo mais notavelmente associado à programação orientada a objetos. No entanto, a linguagem é multifacetada, permitindo abordagens imperativas, funcionais e procedural. Essa flexibilidade permite aos desenvolvedores escolherem o paradigma que melhor se adapta ao problema em questão, contribuindo para a versatilidade da linguagem em diferentes contextos de desenvolvimento.

Em relação aos seus prós, Python é reconhecido pela sua legibilidade e sintaxe limpa, fatores que facilitam a manutenção e colaboração em projetos. A vasta biblioteca padrão e a extensa comunidade de desenvolvedores oferecem suporte para uma ampla gama de tarefas, desde desenvolvimento web até análise de dados.

A sintaxe clara e a legibilidade do Python tornam o desenvolvimento de algoritmos de IA mais acessível, permitindo que os desenvolvedores foquem na lógica e nos conceitos subjacentes, em vez de se perderem em detalhes complexos de implementação. Além disso, a vasta biblioteca padrão e o ecossistema robusto de bibliotecas de terceiros, como NumPy, Pandas e Scikit-learn, oferecem ferramentas poderosas para manipulação de dados, processamento matemático e implementação eficiente de algoritmos de aprendizado de máquina.

Outro ponto crucial é a popularidade e o suporte contínuo que Python recebe na comunidade de ciência de dados e IA. A abundância de frameworks e bibliotecas especializadas, como TensorFlow, PyTorch e scikit-learn, todos com interfaces amigáveis, consolida a posição da linguagem como uma escolha estratégica para desenvolvedores e pesquisadores em IA. A combinação de acessibilidade, riqueza de ferramentas e apoio da comunidade faz de Python uma linguagem ideal para a concepção, prototipagem e implementação eficiente de soluções de inteligência artificial.

\subsection{TensorFlow}

TensorFlow é uma poderosa biblioteca de código aberto desenvolvida pelo Google, projetada para facilitar a implementação de algoritmos de aprendizado de máquina e redes neurais. Destaca-se como uma das ferramentas mais populares e abrangentes no campo da inteligência artificial (IA) e aprendizado de máquina (ML). Seu nome deriva da estrutura subjacente de dados que representa fluxos multidimensionais, essenciais para operações matemáticas eficientes, fundamentais no contexto do aprendizado profundo.

A principal força do TensorFlow reside em sua flexibilidade e suporte para diversas plataformas, possibilitando o desenvolvimento e treinamento de modelos em diferentes ambientes, desde dispositivos móveis até sistemas distribuídos em larga escala. Adotando uma abordagem simbólica, o TensorFlow permite que os desenvolvedores definam e otimizem modelos complexos de maneira eficiente, com ênfase especial em redes neurais profundas. Essa flexibilidade é complementada por uma comunidade ativa e uma vasta documentação, facilitando a curva de aprendizado para novos usuários.

Entre os pontos positivos do TensorFlow, destaca-se a sua escalabilidade e eficácia no treinamento de modelos em grandes conjuntos de dados. Sua integração com unidades de processamento gráfico (GPUs) e unidades de processamento tensorial (TPUs) acelera o processamento, tornando-o ideal para projetos de aprendizado profundo.

\section{Outros Apoios Tecnológicos}
\label{sec:demais}

Seguem as demais ferramentas que auxiliam na na comunicação entre os envolvidos no projeto, bem como na escrita dessa monografia.

\subsection{Ferramentas de Comunicação}

Para comunicação entre autor e orientadores, durante a elaboração desse trabalho, destacam-se: 

\begin{itemize}
	\item O Microsoft Teams, para reuniões virtuais sobre o andamento do trabalho, e 
	\item O Whatsapp, para trocas de mensagens e avisos rápidos.
\end{itemize}

\subsection{Ferramentas de Escrita da Monografia}

Para escrita da monografia, destaca-ase o uso do LaTeX, sendo este uma linguagem de marcação para escrita de monografias comumente utilizada. Como principal vantagem tem-se a escrita 
em forma de texto simples, sem haver tanta preocupação com a formatação durante a escrita do texto. Adicionalmente foi utilizado o Overleaf para escrita do texto no formato LaTeX 
e organização das pastas, além de ser utilizado para compilar a monografia em formato PDF.
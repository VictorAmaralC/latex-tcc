\begin{resumo}[Abstract]
 \begin{otherlanguage*}{english}
  The growing focus on data privacy in the digital age has driven the adoption of innovative approaches to training artificial intelligence (AI) models, particularly in scenarios where data is decentralized. Federated learning (FL) has emerged as a promising solution for training AI models in decentralized environments while preserving user data privacy. This work aims to explore the potential of federated learning to overcome the challenges associated with training AI models on distributed data and ensuring privacy protection. The adopted methodology involved a practical and comparative approach. Initially, strategies of federated learning and privacy-preserving techniques were conceptualized, as well as the security techniques necessary to protect sensitive data. The study also addressed the main challenges of using decentralized data. Next, centralized learning models were implemented using the Fashion-MNIST dataset with different training techniques such as Convolutional Neural Networks (CNN), shallow Multilayer Perceptrons (MLP), and deep Multilayer Perceptrons, which were then compared to a federated learning model using a convolutional neural network technique to train each device that would be aggregated into the global model. Training and performance evaluation were conducted using TensorFlow Federated and Google Colab, focusing on metrics such as accuracy, loss, and top-3 accuracy. The results showed that centralized models, particularly the CNN, achieved superior performance in terms of accuracy and computational efficiency compared to the federated model. However, the federated model, while facing challenges with longer training times and increased communication complexity between devices, showed promising results in privacy preservation, standing out in scenarios where data decentralization is essential. The difficulties encountered in developing the federated model highlighted that, despite competitive performance, federated learning requires additional optimizations to improve computational efficiency. In conclusion, the study highlights the feasibility of federated learning as an approach to maintaining privacy in decentralized scenarios, but also emphasizes the need for further research to overcome its limitations and optimize its practical application.

    \vspace{\onelineskip}
 
    \noindent 
    \textbf{Keywords:} Federated learning, data privacy, machine learning, CNN, data decentralization, TensorFlow Federated.
 \end{otherlanguage*}
\end{resumo}

\begin{resumo}[Abstract]
 \begin{otherlanguage*}{english}
  The increasing focus on data privacy in the digital age has driven the adoption of innovative approaches to training artificial intelligence (AI) models, particularly in scenarios where data is decentralized. Federated Learning (FL) has emerged as a promising solution for training AI models in decentralized environments while maintaining user data privacy. This work aims to explore the potential of federated learning to overcome the challenges associated with training AI models on distributed data and ensuring privacy protection. To achieve this goal, the research was guided by the question: "What are the most significant challenges and limitations associated with the practical implementation of FL in decentralized scenarios, particularly in terms of computational efficiency, device communication, and adaptation to different types of data?". The adopted methodology involved a practical and comparative approach. Initially, federated learning strategies and privacy-preserving techniques were conceptualized, along with the necessary security measures to protect sensitive data. The study also addressed the main challenges of using decentralized data. Subsequently, a federated learning model was implemented using the MNIST dataset and compared with centralized models such as Convolutional Neural Networks (CNNs), Multilayer Perceptrons (MLPs), and Deep Neural Networks (DNNs). Training and performance evaluation were conducted using TensorFlow Federated and Google Colab, with a focus on analyzing metrics such as accuracy, loss, and top-3 accuracy. The results demonstrated that while federated learning provides an effective solution for privacy preservation, it presents significant challenges, such as increased training time and complexity in device communication. Comparison with centralized models revealed that, despite competitive performance, federated learning requires additional optimizations to enhance computational efficiency and adaptability to different types of data. In conclusion, the work highlights the feasibility of federated learning as an approach to maintaining privacy in decentralized scenarios, but also underscores the need for further research to address its limitations and optimize its practical application.


    \vspace{\onelineskip}
 
    \noindent 
    \textbf{Keywords:} Artificial intelligence. Federated Learning. Machine Learning. Data security.
 \end{otherlanguage*}
\end{resumo}

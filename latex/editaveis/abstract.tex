\begin{resumo}[Abstract]
 \begin{otherlanguage*}{english}
   With the growing use of artificial intelligence models in people's daily lives and the associated decentralization of data from various devices such as smartphones, IoT sensors, and local servers, training these models poses substantial challenges in terms of security and privacy. In this context, Federated Learning (FL) emerges as a promising approach for training AI models in a distributed and collaborative manner. We will analyze how FL addresses the inherent challenges in its application in this context, challenges such as data heterogeneity, variability in the quality of network connections, and ethical considerations related to access and use of sensitive information are critical aspects to be addressed. The analysis of these challenges will be grounded in research that highlights the practical and ethical difficulties associated with implementing FL in decentralized scenarios. Lastly, we will address the preservation of users' privacy, exploring the techniques and security measures necessary to ensure that distributed data remains confidential during federated training. In summary, this work seeks to offer a comprehensive and in-depth view of the applicability of FL in training AI models in decentralized environments, highlighting not only the advantages but also the challenges and solutions necessary for preserving users' privacy.


    \vspace{\onelineskip}
 
    \noindent 
    \textbf{Keywords:} Artificial intelligence. Federated Learning. Machine Learning. Data security.
 \end{otherlanguage*}
\end{resumo}

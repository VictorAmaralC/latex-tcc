\chapter[Metodologia]{Metodologia}
\addcontentsline{toc}{chapter}{Metodologia}
\label{chap:metodologia}

\section{Tipo de Estudo}

Este trabalho adota uma abordagem de revisão de literatura integrativa combinada com um estudo experimental. A revisão de literatura tem como objetivo sintetizar o estado atual da pesquisa sobre aprendizado federado (Federated Learning – FL), focando em sua aplicabilidade no treinamento de modelos de IA com dados descentralizados, preservando a privacidade dos usuários. A revisão será baseada em artigos científicos, teses, dissertações e relatórios de conferências. O estudo experimental complementará a revisão, implementando um modelo de aprendizado federado utilizando o framework TensorFlow Federated (TFF) para treinar uma rede neural com o conjunto de dados MNIST distribuído. Esse estudo permitirá comparar o desempenho de um modelo federado com um modelo centralizado, verificando as vantagens e desvantagens práticas do aprendizado federado em termos de precisão e privacidade.

\section{Fonte de Dados}

A coleta de dados será dividida em duas partes: fontes teóricas e fontes práticas. Para as fontes teóricas, serão consultados artigos científicos, livros, dissertações e teses indexados em bases de dados acadêmicas, como IEEE Xplore, ACM Digital Library, SpringerLink, e Google Scholar. A seleção desses artigos seguirá critérios detalhados mais adiante. Já a fonte de dados prática será o conjunto de dados MNIST, amplamente utilizado em experimentos de aprendizado de máquina para o reconhecimento de dígitos manuscritos. O MNIST será aplicado tanto no treinamento centralizado quanto no treinamento federado, permitindo uma comparação direta entre os métodos.

\section{Critérios de Inclusão e Exclusão}

Os critérios para a seleção de artigos científicos na revisão de literatura seguirão a metodologia PRISMA (Preferred Reporting Items for Systematic Reviews and Meta-Analyses), com base nos seguintes critérios:

Critérios de Inclusão:
\begin{itemize}
    \item Artigos que tratam diretamente do aprendizado federado e suas aplicações em IA.
    \item Estudos que discutem a privacidade de dados em cenários de treinamento distribuído.
    \item Publicações entre 2015 e 2024, dado o crescimento recente da pesquisa em aprendizado federado.
    \item Artigos revisados por pares e disponíveis em bases de dados confiáveis (IEEE, ACM, Springer, etc.).
    \item Estudos que utilizam frameworks práticos como TensorFlow Federated.
\end{itemize}

Critérios de Exclusão:
\begin{itemize}
    \item Artigos que não abordam a privacidade de dados ou aprendizado federado.
    \item Publicações anteriores a 2015, a menos que sejam referências essenciais ou históricas.
    \item Estudos que não apresentam implementações práticas ou são puramente teóricos sem aplicação clara.
\end{itemize}

\section{Procedimentos de Coleta de Dados}

A coleta de dados teóricos será realizada em três fases principais:
1. Busca inicial: serão utilizados termos de pesquisa como "federated learning", "data privacy in AI", "decentralized machine learning", e "TensorFlow Federated".
2. Análise de relevância: cada artigo será avaliado com base nos títulos e resumos para verificar a aderência aos critérios de inclusão e exclusão.
3. Leitura aprofundada: os artigos selecionados serão lidos na íntegra, com ênfase nos objetivos, metodologia e resultados, focando nas discussões que envolvem aprendizado federado, privacidade e comparações entre modelos centralizados e federados.

Para a parte prática, o conjunto de dados MNIST será particionado de forma a simular um cenário de dados descentralizados, onde os dados serão distribuídos em vários "clientes". Esses clientes terão acesso a subconjuntos dos dados completos, conforme a configuração típica de um sistema de aprendizado federado.

\section{Análise de Dados}

Os dados coletados serão analisados de duas formas principais:

1. Revisão de literatura: será realizada uma análise crítica dos artigos selecionados, categorizando-os de acordo com as abordagens de privacidade, frameworks utilizados, resultados de desempenho e propostas para melhorias no aprendizado federado. Será feita uma comparação entre os diferentes métodos de preservação de privacidade, identificando lacunas de pesquisa.

2. Estudo experimental: os dados práticos serão analisados com base nos resultados de desempenho dos modelos treinados. As métricas de avaliação incluirão acurácia, precisão, sensibilidade, F1-score e tempo de treinamento. Ferramentas como TensorFlow e TensorFlow Federated serão utilizadas para construir os modelos e calcular essas métricas. A análise comparativa entre o modelo centralizado e o federado será realizada para verificar as vantagens e desvantagens de cada abordagem, com foco em privacidade e desempenho.

Esta metodologia permite uma análise abrangente do tema, garantindo que o estudo aborde tanto a teoria quanto a prática do aprendizado federado, enquanto investiga os desafios de manter a privacidade dos dados em cenários descentralizados.
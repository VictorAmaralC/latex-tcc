\begin{resumo}

O crescente foco na preservação da privacidade dos dados na era digital tem impulsionado a adoção de abordagens inovadoras no treinamento de modelos de inteligência artificial (IA), especialmente em cenários onde os dados são descentralizados. O aprendizado federado (AF) emergiu como uma solução promissora para treinar modelos de IA em ambientes descentralizados, mantendo a privacidade dos dados dos usuários. Este trabalho tem como objetivo explorar o potencial do aprendizado federado para superar os desafios associados ao treinamento de modelos de IA em dados distribuídos e garantir a proteção da privacidade. Para alcançar esse objetivo, a pesquisa foi guiada pela questão: "Quais são os desafios e limitações mais significativos associados à implementação prática do AF em cenários descentralizados, especialmente em termos de eficiência computacional, comunicação entre dispositivos e adaptação a diferentes tipos de dados?". A metodologia adotada envolveu uma abordagem prática e comparativa. Inicialmente, foram conceituadas as estratégias de aprendizado federado e as técnicas de preservação da privacidade, além das técnicas de segurança necessárias para proteger dados sensíveis. O estudo também abordou os principais desafios do uso de dados descentralizados. Em seguida, foi implementado um modelo de aprendizado federado utilizando a base de dados MNIST e comparado com modelos centralizados, como Redes Neurais Convolucionais (CNN), Perceptrons Multicamadas (MLP) e Redes Neurais Profundas (DNN). O treinamento e a avaliação de desempenho foram conduzidos utilizando TensorFlow Federated e Google Colab, com foco na análise de métricas como acurácia, perda e acurácia top-3. Os resultados demonstraram que, embora o modelo de aprendizado federado ofereça uma solução eficaz para a preservação da privacidade, ele apresenta desafios significativos, como maior tempo de treinamento e complexidade na comunicação entre dispositivos. A comparação com os modelos centralizados evidenciou que, apesar de um desempenho competitivo, o aprendizado federado requer otimizações adicionais para melhorar a eficiência computacional e a adaptabilidade a diferentes tipos de dados. Em conclusão, o trabalho destaca a viabilidade do aprendizado federado como uma abordagem para manter a privacidade em cenários descentralizados, mas também ressalta a necessidade de mais pesquisas para superar suas limitações e otimizar sua aplicação prática.

\vspace{\onelineskip}
    
\noindent
\textbf{Palavras-chave:} Inteligência artificial. Aprendizado Federado. Aprendizado de Máquina. Segurança. 
\end{resumo}
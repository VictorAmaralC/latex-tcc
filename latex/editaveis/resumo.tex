\begin{resumo}

O crescente foco na preservação da privacidade dos dados na era digital tem impulsionado a adoção de abordagens inovadoras no treinamento de modelos de inteligência artificial (IA), especialmente em cenários onde os dados são descentralizados. O aprendizado federado (AF) emergiu como uma solução promissora para treinar modelos de IA em ambientes descentralizados, mantendo a privacidade dos dados dos usuários. Este trabalho tem como objetivo explorar o potencial do aprendizado federado para superar os desafios associados ao treinamento de modelos de IA em dados distribuídos e garantir a proteção da privacidade. A metodologia adotada envolveu uma abordagem prática e comparativa. Inicialmente, foram conceituadas as estratégias de aprendizado federado e as técnicas de preservação da privacidade, além das técnicas de segurança necessárias para proteger dados sensíveis. O estudo também abordou os principais desafios do uso de dados descentralizados. Em seguida, foram implementado modelos de aprendizado centralizado utilizando a base de dados Fashion-MNIST com diferentes técnicas de treinamento como Redes Neurais Convolucionais (CNN), Perceptrons Multicamadas (MLP) rasas e Perceptrons Multicamadas profundas que depois foram comparados com um modelo de aprendizado federado utilizando a técnica rede neural convolucional para efetuar o treinamento de de cada dispositivo a ser agregado no modelo global. O treinamento e a avaliação de desempenho foram conduzidos utilizando TensorFlow Federated e Google Colab, com foco na análise de métricas como acurácia, perda e acurácia top-3. Os resultados obtidos mostraram que os modelos centralizados, especialmente a CNN, alcançaram um desempenho superior em termos de acurácia e eficiência computacional em relação ao modelo federado. No entanto, o modelo federado, embora tenha enfrentado desafios com tempos de treinamento mais longos e maior complexidade na comunicação entre dispositivos, apresentou resultados promissores na preservação da privacidade, destacando-se em cenários onde a descentralização dos dados é fundamental. As dificuldades ao desenvolver o modelo federado evidenciaram que, apesar de um desempenho competitivo, o aprendizado federado requer otimizações adicionais para melhorar a eficiência computacional. Em conclusão, o trabalho destaca a viabilidade do aprendizado federado como uma abordagem para manter a privacidade em cenários descentralizados, mas também ressalta a necessidade de mais pesquisas para superar suas limitações e otimizar sua aplicação prática.

\vspace{\onelineskip}
    
\noindent
\textbf{Palavras-chave:} Inteligência artificial. Aprendizado Federado. Aprendizado de Máquina. Segurança.
\end{resumo}
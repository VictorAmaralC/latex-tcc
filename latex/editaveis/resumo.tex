\begin{resumo}
 Com a crescente utilização de modelos de inteligência artificial no dia a dia das pessoas e associada a isso a descentralização dos dados, provenientes de dispositivos diversos como smartphones, sensores IoT e servidores locais, o treinamento desses modelos apresenta desafios substanciais em termos de segurança e privacidade. Nesse contexto, o Aprendizado Federado (AF) emerge como uma abordagem promissora para o treinamento de modelos de IA de forma distribuída e colaborativa. Analisaremos como o AF lida com os desafios inerentes a sua aplicação nesse contexto, desafios como Heterogeneidade de dados, variabilidade na qualidade das conexões de rede e considerações éticas relacionadas ao acesso e uso de informações sensíveis são aspectos críticos a serem abordados. A análise desses desafios será embasada em pesquisas que evidenciam as dificuldades práticas e éticas associadas à implementação do AF em cenários descentralizados. Por último abordaremos a preservação da privacidade dos usuários, exploraremos as técnicas e medidas de segurança necessárias para garantir que os dados distribuídos permaneçam confidenciais durante o treinamento federado. Em síntese, este trabalho busca oferecer uma visão abrangente e aprofundada sobre a aplicabilidade do AF no treinamento de modelos de IA em ambientes descentralizados, destacando não apenas as vantagens, mas também os desafios e soluções necessárias para a preservação da privacidade dos usuários.

 \vspace{\onelineskip}
    
 \noindent
 \textbf{Palavras-chave:} Inteligência artificial. Aprendizado Federado. Aprendizado de Máquina. Segurança. 
\end{resumo}